\newglossaryentry{gl-devops}
{
    name=DevOps,
    description={Ein Prozessverbesserungsansatz, der verschiedene Werkzeuge und Prozesse einsetzt, um bei den Bereichen der Entwicklung, Qualitätssicherung und dem IT Betrieb für eine automatisierte Zusammenarbeit zu sorgen}
}

\newglossaryentry{gl-onpremise}{
  name=On-Premise,
  description={Eine Applikation, die nur lokal ausgeführt und benutzt werden kann}
}

% \newglossaryentry{gl-orm}{
%     name=ORM,
%     description={Eine code Bibliothek, welche Daten, gespeichert als relationale Tabellen automatisiert in Objekte transferiert, die regelmäßig in Applikationscode genutzt werden}
% }

\newglossaryentry{gl-e2e}{
    name=E2E,
    description={Eine Testmethode, um den Verlauf einer Applikation von Anfang bis Ende zu testen. Dadurch sollen mithilfe von simulierten aber echten Nutzerszenarien die Daten- und Systemintegration mit dessen Komponenten validiert werden}
}

\newglossaryentry{gl-loadbalancer}{
    name=Load Balancer,
    description={Ein Gerät, welches ankommende Netzwerkanfragen zwischen Servern verteilt, sodass kein Server mit Anfragen überlastet wird}
}

\newglossaryentry{gl-solid}{
  name=SOLID,
  description={Prinzipien der objektorientierten Entwicklung. Diese Prinzipien sind:
  \begin{itemize}
    \item Single-Responsibility-Prinzip
    \item Open-closed-Prinzip
    \item Liskov-substitution-Prinzip
    \item Interface-segregation-Prinzip
    \item Dependency-inversion-Prinzip
  \end{itemize}
  Die Anfangsbuchstaben der Prinzipien bilden das Akronym \emph{SOLID}}
}

\newglossaryentry{gl-domaene}{
  name=Domäne,
  description={Ein Arbeitsbereich, Anwendungsgebiet oder ein Problemfeld}
}

% \newglossaryentry{gl-frontend}{
%   name=Frontend,
%   description={Code und Architektur, der die grafische Benutzeroberfläche definiert}
% }

% \newglossaryentry{gl-backend}{
%   name=Backend,
%   description={Code und Architektur, der die Geschäftslogik eines Systems widerspiegelt}
% }

% \newglossaryentry{gl-crud}{
%   name=CRUD,
%   description={Die üblichen Operationen zum Verwalten von Datenbanken:
%   \begin{itemize}
%     \item Create
%     \item Read
%     \item Update
%     \item Delete
%   \end{itemize}
%   Die Anfangsbuchstaben der Operationen bilden das Akronym \emph{CRUD}
%   }
% }